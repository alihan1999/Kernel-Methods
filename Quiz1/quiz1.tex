\documentclass[12pt]{article}
\usepackage[english]{babel}
\usepackage{amsmath}
\usepackage{amsthm}
\usepackage{amssymb}
\usepackage{amsfonts}
\title{HDP Notes and Exercise Soultions}
\author{Holden Caulfield}
\newtheorem{theorem}{Theorem}
\newtheorem{remark}{Remark}[theorem]
\newtheorem{ex}{Problem}
\usepackage{enumitem}
\usepackage{graphicx}
\usepackage{tcolorbox}
\usepackage{caption}
\usepackage{xcolor}
\usepackage{tikz}
\tcbuselibrary{skins}
\tcbuselibrary{breakable}

\newtcolorbox{bx}[1]{skin=bicolor,breakable,leftrule=1mm,toprule=0mm,bottomrule=0mm,rightrule=0mm,colbacklower=red!15,colback=gray!15,sharp corners,colframe=red}

\begin{document}
	\centering	\section*{Quiz 1 Solution}
	
	\begin{bx}
		
	\begin{ex}
For each given kernel, check whether they are positive definite
\begin{enumerate}
	\item $\chi=(-1,1), K(x,y)=\frac{1}{1-xy}$
	\item $\chi=\mathbb{N}, K(x,y)=2^{xy}$
	\item $\chi=\mathbb{R}_{+}, K(x,y)=\log(1+xy)$
	\item $\chi=\mathbb{R}, K(x,y)=\exp(-|x-y|^2)$
	\item $\chi=\mathbb{R}_{+}, K(x,y)=\max(x,y)$
	\item $\chi=\mathbb{R}_{+}, K(x,y)=\min(x,y)$
	\item $\chi=\mathbb{R}_{+}, K(x,y)=\frac{\min(x,y)}{\max(x,y)}$
	\item $\chi=\mathbb{N}, K(x,y)=\gcd(x,y)$
	\item $\chi=\mathbb{N}, K(x,y)=\mathrm{lcm}(x,y)$
	\item $\chi=\mathbb{N}, K(x,y)=\frac{\mathrm{\gcd(x,y)}}{lcm(x,y)}$
\end{enumerate}
	\end{ex}
\tcblower
\begin{enumerate}
\item 
Set $f(x,y)=xy$. it is easy to see that $f$ is a PD kernel.
Set $K_n(x,y)=\sum_{r=0}^{n}f(x,y)^r$.
Since $|f(x,y)|< 1$,$\lim\limits_{n\rightarrow\infty}K_n(x,y)=K(x,y)$.
We also know that each $K_n$ is a PD kernel (using the basic facts about PD kernels), which makes the limit i.e. $K$ a PD kernel.

\item As we know, for any PD kernel $K^\prime$,$e^{K^\prime}$ is also a PD kernel. Again, setting $f(x,y)=xy$, we have $2^{f(x,y)}=e^{\ln(2)f(x,y)}$. We know that $\ln(2)>0$ and that $f(x,y)$ is a PD kernel, which makes $\ln(2)f(x,y)$ and therefore $e^{\ln(2)f(x,y)}$ PD kernels.

\item
setting $X=\{1,3\}$, the matrix of $K$ would have a negative determinant, therefore it can not be positive semi-definite, hence $K$ is not a PD kernel (Mercer's theorem).

\item Since the domain is $\mathbb{R}$, we are allowed to set $|x-y|^2=(x-y)^2$. We have:
$$
	\exp(-|x-y|^2)=\exp(-(x-y)^2)=\exp(2xy)\exp(-x^2)\exp(-y^2)
$$
Set $f(x,y)=\exp(-x^2)\exp(-y^2)$.
It is easy to see that $f$ is a PD kernel. Further, $\exp(2xy)$ is also a PD kernel, and the multiplication of two PD kernels is a PD kernel, which makes $K$ a PD kernel.

\item Set $X=\{1,2\}$. The matrix of $K$ with respect to $X$ has a negative determinant, hence it is not positive semi-definite and $K$ is not a PD kernel.

\item Set $X=\{x_1,\dots,x_n\}$ and assume that $\max\limits_{i} x_i=N$.

 Take
 $V=[v_1,\dots,v_n]^\top, v_i\in \mathbb{R}^N$, where $v_{ij}=1$ for $j\in[x_i]$ and $v_{ij}=0$ elsewhere.
 
 The matrix of $K$ with respect to $X$ is $VV^\top$, which is the Gram matrix of $v_i$'s. We know that all Gram matrices are positive semi-definite, which implies that $K$ is a PD kernel according to Mercer's theorem.
 
 \item We can write $\max(x,y)=\frac{1}{\min(1/x,1/y)}$, so we have:
 \[
 \frac{\min(x,y)}{\max(x,y)}=\min(x,y).\min(1/x,1/y)
 \]
 As we showed above, $\min$ is a PD kernel, which implies that $K$ is also a PD kernel.
 
 
 \item Represent the prime factorization of $x$ and $y$ with vectors $X,Y \in \mathbb{R}^P$ where $P$ is the largest prime that appears in the factorization of $x$ and $y$.
 Set $X_p=p^j$ if $p$ appears in the prime factorization of $x$ with power $j$, and set $X_i=1$ elsewhere, and define $Y$ in the same manner.
 
 We can now define $\gcd(x,y)=\prod_{k=1}^{P}\min(X_k,Y_k)$. With this interpretation, we can see that $K$ is a PD kernel, since $\min$ is a PD kernel and the multiplication of PD kernels is also a PD kernel. 
 
 \item Set $X=\{1,2\}$. The matrix of $K$ with respect to $X$ has a negative determinant and is not positive semi-definite, which means that $K$ is not a PD kernel.
 
 \item we know that $\mathrm{lcm}(x,y)=\frac{xy}{\gcd(x,y)}$. We have:
 \[
 \frac{\gcd(x,y)}{\mathrm{lcm}(x,y)}=\frac{\gcd^2(x,y)
 }{xy}
 \]
 Setting $f(x,y)=\frac{1}{x}.\frac{1}{y}$, it is easy to check that $f$ is PD, which implies that $K$ is also PD.
\end{enumerate}
	\end{bx}
	
	
\end{document}